\documentclass[12pt]{article}

% Common packages
\usepackage[utf8]{inputenc}      % UTF-8 input encoding
\usepackage[T1]{fontenc}         % Better font encoding
\usepackage{lmodern}             % Latin Modern font
\usepackage{amsmath, amssymb}    % Math symbols and environments
\usepackage{geometry}            % Page layout
\geometry{a4paper, margin=1in}
\usepackage{graphicx}            % For including images
\usepackage{hyperref}            % Clickable links and PDF metadata

% Document metadata
\title{Theory of 2D-Rockets basically}
\author{Zhao H.Q.}
\date{\today}

\begin{document}

\maketitle

\section{Assumption}
1.2Dimensional motion: Our rockets only consider 2Dimension,x-z flat (The z-axis points upward, and the x-axis is horizontal.)\\
2.rigid body: The rocket is treated as a rigid body, meaning it does not deform during flight.\\
3.Translation and rotation: The rocket's motion includes both translation (movement through space) and rotation (spinning around its center of mass).\\
4.Aerodynamic force only acts as drag.\\
5.Gravity acts downward along the z-axis with a constant acceleration of \( g = 9.81 \, \text{m/s}^2 \).\\
6.constant air density of \( \rho = 1.225 \, \text{kg/m}^3 \) (sea level standard conditions).\\

\section{Coordinate system and state variables}

\subsection{1.Definition of system coordinate }
1.Inertial frame of reference: origin at launch point
2.

\end{document}